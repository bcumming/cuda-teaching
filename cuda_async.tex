%++++++++++++++++++++++++++++++++
\cscschapter{Concurrency}
%++++++++++++++++++++++++++++++++

%%%%%%%%%%%%%%%%%%%%%%%%%%%%%%%%%%%%%%%%%%%%
\begin{frame}[fragile]{}
%%%%%%%%%%%%%%%%%%%%%%%%%%%%%%%%%%%%%%%%%%%%
    \begin{info}{concurrency}
        \emph{Concurrency} is the ability to perform multiple CUDA operations simultaneously
        \begin{itemize}
            \item CUDA kernels
            \item copying from host to device
            \item copying from device to host
            \item operations on the host CPU
        \end{itemize}
    \end{info}

    \begin{info}{concurrency enables}
        \begin{itemize}
            \item both CPU and GPU can work at the same time
            \item multiple tasks can be run on GPU simultaneously
            \item we can overlap communication and computation
        \end{itemize}
    \end{info}

\end{frame}

%%%%%%%%%%%%%%%%%%%%%%%%%%%%%%%%%%%%%%%%%%%%
\begin{frame}[fragile]{CPU and GPU operations are asynchronous}
%%%%%%%%%%%%%%%%%%%%%%%%%%%%%%%%%%%%%%%%%%%%
    \begin{columns}[T]
        \begin{column}{0.5\textwidth}
            \begin{codecolumn}{host code}
                \begin{lstlisting}[style=boxcudatiny]
kernel_1<<<...>>>(...);
kernel_2<<<...>>>(...);
host_1(...);
host_2(...);
                \end{lstlisting}
            \end{codecolumn}
        The host:
        \begin{itemize}
            \item launches the two CUDA kernels
            \item then executes host calls sequentially 
        \end{itemize}
        The GPU:
        \begin{itemize}
            \item executes asynchronously to host
            \item executes kernels sequentially
        \end{itemize}
        \end{column}
        \begin{column}{0.5\textwidth}
            \includegraphics[width=\textwidth]{./images/async_null.pdf}
        \end{column}
    \end{columns}
\end{frame}

%%%%%%%%%%%%%%%%%%%%%%%%%%%%%%%%%%%%%%%%%%%%
\begin{frame}[fragile]{}
%%%%%%%%%%%%%%%%%%%%%%%%%%%%%%%%%%%%%%%%%%%%
    The CUDA language and runtime libraries provide mechanisms for coordinating asynchronous GPU execution

    \begin{itemize}
        \item \emph{CUDA streams} can concurrently run independent kernels and memory transfers
        \item \emph{CUDA events} can be used to synchronize streams and query the status of kernels and transfers
    \end{itemize}

\end{frame}

%%%%%%%%%%%%%%%%%%%%%%%%%%%%%%%%%%%%%%%%%%%%
\begin{frame}[fragile]{}
%%%%%%%%%%%%%%%%%%%%%%%%%%%%%%%%%%%%%%%%%%%%
    \begin{info}{streams}
        A CUDA stream is is a sequence of operations that execute in \emph{issue order} on the GPU
        \begin{itemize}
            \item CUDA operations are kernels and copies between host and device memory spaces
        \end{itemize}
    \end{info}

    \begin{info}{streams and concurrency}
        \begin{itemize}
            \item operations in different streams \emph{may} run concurrently
            \begin{itemize}
                \item there have to be sufficient resources on the GPU (registers, shared memory, blocks, etc)
            \end{itemize}
            \item operations in the same stream \emph{are} executed sequentially
            \item if no stream is specified, all kernels are launched in the default stream
        \end{itemize}
    \end{info}

\end{frame}

%%%%%%%%%%%%%%%%%%%%%%%%%%%%%%%%%%%%%%%%%%%%
\begin{frame}[fragile]{}
%%%%%%%%%%%%%%%%%%%%%%%%%%%%%%%%%%%%%%%%%%%%
    \begin{info}{managing streams}
        A stream is represented using a \lst{cudaStream_t} type
        \begin{itemize}
            \item \lst{cudaStreamCreate(cudaStream_t* s)} and \lst{cudaStreamDestroy(cudaStream_t s)} can be used to create and free CUDA streams respectively
            \item To launch a kernel on a stream specify the stream id as a fourth parameter to the launch syntax \\
                \begin{center} \lst{kernel<<<grid_dim, block_dim, shared_size, stream>>>(...)} \end{center}
            \item the default CUDA stream is the \lst{NULL} stream, or stream 0 (\lst{cudaStream_t} is an integer)
        \end{itemize}
    \end{info}

    \begin{code}{basic cuda stream useage}
        \begin{lstlisting}[style=boxcudatiny]
// create stream
cudaStream_t stream;
cudaStreamCreate(&stream);
// launch kernel in stream
my_kernel<<<grid_dim, block_dim, shared_size, stream>>>(..)
// release stream when finished
cudaStreamDestroy(stream);
        \end{lstlisting}
\end{code}

\end{frame}

%%%%%%%%%%%%%%%%%%%%%%%%%%%%%%%%%%%%%%%%%%%%
\begin{frame}[fragile]{Asynchronous device execution example}
%%%%%%%%%%%%%%%%%%%%%%%%%%%%%%%%%%%%%%%%%%%%
    \begin{columns}[T]
        \begin{column}{0.45\textwidth}
            \begin{codecolumn}{host code}
                \begin{lstlisting}[style=boxcudatiny]
kernel_1<<<,,,stream_1>>>();
kernel_2<<<,,,stream_2>>>();
kernel_3<<<,,,stream_1>>>();
                \end{lstlisting}
            \end{codecolumn}
            \begin{itemize}
                \item \footnotesize \lst{kernel_1} and \lst{kernel_2} are serialized in \lst{stream_1}
                \item \lst{kernel_2} can run asynchronously in \lst{stream_2}
                \item note that \lst{kernel_2} will only run concurrently if there are sufficient resources available on the GPU, i.e. if \lst{kernel_1} is not using all of the SMXs.
            \end{itemize}
        \end{column}
        \begin{column}{0.6\textwidth}
            \includegraphics[width=\textwidth]{./images/async_two_streams.pdf}
        \end{column}
    \end{columns}
\end{frame}

%%%%%%%%%%%%%%%%%%%%%%%%%%%%%%%%%%%%%%%%%%%%
\begin{frame}[fragile]{}
%%%%%%%%%%%%%%%%%%%%%%%%%%%%%%%%%%%%%%%%%%%%

    \begin{info}{asynchronous copy}
        \centering \lst{cudaMemcpyAsync(*dst, *src, count, kind, cudaStream_t stream = 0);}
        \begin{itemize}
            \item takes an additional parameter stream, which is 0 by default
            \item returns immediately after initiating copy
            \begin{itemize}
                \item host can do work while copy is performed
                \item only if \emph{pinned memory} is used
            \end{itemize}
            \item copies in the same direction (i.e. H2D or D2H) are serialized
            \begin{itemize}
                \item copies in opposite directions are concurrent if in different streams
            \end{itemize}
        \end{itemize}
    \end{info}

\end{frame}

%%%%%%%%%%%%%%%%%%%%%%%%%%%%%%%%%%%%%%%%%%%%
\begin{frame}[fragile]{}
%%%%%%%%%%%%%%%%%%%%%%%%%%%%%%%%%%%%%%%%%%%%
    \begin{info}{what is pinned memory?}
        Pinned memory (or page-locked) memory will not be paged out to disk when memory runs low
        \begin{itemize}
            \item the GPU can safely remotely read/write the memory directly without host involvement
            \item only use for transfers, because you run out of memory
        \end{itemize}
    \end{info}

    \begin{info}{managing pinned memory}
        \centering \lst{cudaMallocHost(**ptr, size);} and \lst{cudaFreeHost(*ptr);}
        \begin{itemize}
            \item allocate and free pinned memory (\lst{size} is in bytes).
        \end{itemize}
    \end{info}

\end{frame}

%%%%%%%%%%%%%%%%%%%%%%%%%%%%%%%%%%%%%%%%%%%%
\begin{frame}[fragile]{}
%%%%%%%%%%%%%%%%%%%%%%%%%%%%%%%%%%%%%%%%%%%%
    \begin{info}{Asynchronous copy example: streaming workloads}
        Computations that can be performed independently, e.g. our \axpy example:
        \begin{itemize}
            \item data in host memory has to be copied to the device, and the result copied back after the kernel is computed.
            \item we can overlap the copies with the kernel calls by breaking the data into chunks.
        \end{itemize}
    \end{info}
    \includegraphics[width=\textwidth]{./images/overlap.pdf}
\end{frame}

%%%%%%%%%%%%%%%%%%%%%%%%%%%%%%%%%%%%%%%%%%%%
\begin{frame}[fragile]{}
%%%%%%%%%%%%%%%%%%%%%%%%%%%%%%%%%%%%%%%%%%%%
    \begin{info}{CUDA events}
        To implement the streaming workload we have to coordinate operations on the GPU.
        CUDA events can be used for this purpose.
        \begin{itemize}
            \item synchronize tasks in different streams, e.g.:
            \begin{itemize}
                \item don't start kernel in kernel stream until data copy stream has finished.
                \item wait until required data has finished copy from host before launching kernel
            \end{itemize}
            \item query status of concurrent tasks
            \begin{itemize}
                \item has kernel finished/started yet?
                \item how long did a kernel take to compute?
            \end{itemize}
        \end{itemize}
    \end{info}
\end{frame}

%%%%%%%%%%%%%%%%%%%%%%%%%%%%%%%%%%%%%%%%%%%%
\begin{frame}[fragile]{}
%%%%%%%%%%%%%%%%%%%%%%%%%%%%%%%%%%%%%%%%%%%%
    \begin{info}{managing events}
        \lst{cudaEventCreate(cudaEvent_t*);} and \lst{cudaEventDestroy(cudaEvent_t);}
            \begin{itemize}
                \item create and free \lst{cudaEvent_t}
            \end{itemize}
        \lst{cudaEventRecord(cudaEvent_t, cudaStream_t_);}
            \begin{itemize}
                \item enqueue an event in a stream
            \end{itemize}
        \lst{cudaEventSynchronize(cudaEvent_t);}
            \begin{itemize}
                \item make host execution wait for event to occur.
            \end{itemize}
        \lst{cudaEventQuery(cudaEvent_t)}
            \begin{itemize}
                \item test if the work before an event in a queue has been completed
            \end{itemize}
        \lst{cudaEventElapsedTime(float*, cudaEvent_t, cudaEvent_t);}
            \begin{itemize}
                \item get time between two events
            \end{itemize}
    \end{info}
\end{frame}

%%%%%%%%%%%%%%%%%%%%%%%%%%%%%%%%%%%%%%%%%%%%
\begin{frame}[fragile]{}
%%%%%%%%%%%%%%%%%%%%%%%%%%%%%%%%%%%%%%%%%%%%
    \begin{code}{using events to time kernel execution}
        \begin{lstlisting}[style=boxcudatiny]
cudaEvent_t start, end;
cudaStream_t stream;
float time_taken;

// initialize the events and streams
cudaEventCreate(&start);
cudaEventCreate(&end);
cudaStreamCreate(&stream);

cudaEventRecord(start, stream); // put start in stream
my_kernel<<<grid_dim, block_dim, 0, stream>>>();
cudaEventRecord(end, stream);   // put end in stream
cudaEventSynchronize(end);      // wait for end to be reached
cudaEventElapsedTime(&time_taken, start, end);

std::cout << "kernel took " << 1000*time_taken << " s\n";

// free resources for events and streams
cudaEventDestroy(start);
cudaEventDestroy(end);
cudaStreamDestroy(stream);
        \end{lstlisting}
    \end{code}
\end{frame}

%%%%%%%%%%%%%%%%%%%%%%%%%%%%%%%%%%%%%%%%%%%%
\begin{frame}[fragile]{}
%%%%%%%%%%%%%%%%%%%%%%%%%%%%%%%%%%%%%%%%%%%%
    \begin{code}{using events to time kernel execution}
        \begin{lstlisting}[style=boxcudatiny]
CudaEvent start, end;
CudaStream stream(true);

auto start = stream.add_event();
my_kernel<<<grid_dim, block_dim, 0, stream.stream()>>>();
auto end = stream.add_event();
end.wait();
auto time_taken = end.time_since(start);

std::cout << "kernel took " << 1000*time_taken << " s\n";
        \end{lstlisting}
    \end{code}
\end{frame}
%%%%%%%%%%%%%%%%%%%%%%%%%%%%%%%%%%%%%%%%%%%%
\begin{frame}[fragile]{Events : copy-kernel synchronization example}
%%%%%%%%%%%%%%%%%%%%%%%%%%%%%%%%%%%%%%%%%%%%
    code
\end{frame}

%%%%%%%%%%%%%%%%%%%%%%%%%%%%%%%%%%%%%%%%%%%%
\begin{frame}[fragile]{Exercises}
%%%%%%%%%%%%%%%%%%%%%%%%%%%%%%%%%%%%%%%%%%%%
    \begin{itemize}
        \item time to visit \lst{util.h} and add helpers for asynchronous copy
        \item just give the event and stream wrappers ``as is''
        \begin{itemize}
            \item walk through their use in the memcpy example
        \end{itemize}
        \item fire up nvprof
    \end{itemize}
\end{frame}

%%%%%%%%%%%%%%%%%%%%%%%%%%%%%%%%%%%%%%%%%%%%
\begin{frame}[fragile]{Asynchronous example: streaming}
%%%%%%%%%%%%%%%%%%%%%%%%%%%%%%%%%%%%%%%%%%%%
    Streaming is a workload where we can break data and work into chunks
    \begin{itemize}
        \item work on one chunk of data while the next chunk is being sent
        \item send each chunk back to host and take next available chunk of work
        \item there has to be enough work in each chunk to hide ...
    \end{itemize}

    Take, for example our axpy example...
    \begin{itemize}
        \item clearly no amount of overlap will help us
        \item note that we get full speed both directions
    \end{itemize}
\end{frame}

%%%%%%%%%%%%%%%%%%%%%%%%%%%%%%%%%%%%%%%%%%%%
\begin{frame}[fragile]{Asynchronous example: streaming}
%%%%%%%%%%%%%%%%%%%%%%%%%%%%%%%%%%%%%%%%%%%%
    Show Newton kernel : lotsa work
\end{frame}

%%%%%%%%%%%%%%%%%%%%%%%%%%%%%%%%%%%%%%%%%%%%
\begin{frame}[fragile]{Streams: rule of thumb}
%%%%%%%%%%%%%%%%%%%%%%%%%%%%%%%%%%%%%%%%%%%%
    Ideally for most workloads you don't want to rely on streams to fill the GPU with work
    \begin{itemize}
        \item a sign that the working set per GPU is not large enough
        \item full concurrency is difficult in practice
        \begin{itemize}
            \item a low-level optimization strategy for the last few \%
        \end{itemize}
        \item this isn't a hard and fast rule
    \end{itemize}

    Streams come into their own for overlapping communication and computation
    \begin{itemize}
        \item possible to transfer data in both directions concurrently with kernels execution
    \end{itemize}
\end{frame}


%%%%%%%%%%%%%%%%%%%%%%%%%%%%%%%%%%%%%%%%%%%%
\begin{frame}[fragile]{Exercises}
%%%%%%%%%%%%%%%%%%%%%%%%%%%%%%%%%%%%%%%%%%%%
    Overlapping work for Newton
\end{frame}
