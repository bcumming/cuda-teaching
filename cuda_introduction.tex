% CHAPTER SLIDE
\cscschapter{Introduction}

%%%%
\begin{frame}[fragile]{Introduction to CUDA}
    \begin{info}{The plan}
        \begin{itemize}
            \item learn about the GPU programming model
            \item implement CUDA kernels for simple linear algebra
            \item learn how to orchestrate asynchronous computaion and communication on GPU
            \item a 2D stencil application
        \end{itemize}
    \end{info}

    \begin{info}{Prerequisites}
        \begin{itemize}
            \item I assume C++ knowledge
            \begin{itemize}
                \item I will be using C++11 (the bits that make C++ easier!)
                \item Fortran users consider working with a C++ user
            \end{itemize}
            \item No GPU or graphics experience required
        \end{itemize}
    \end{info}

\end{frame}

%%%%
\begin{frame}[fragile]{What is CUDA?}
    \begin{info}{A superset of C++}
        \begin{itemize}
            \item write CPU code using C++ (C++11 since CUDA 6.5)
            \item write kernels to run on GPU using new keywords
            \item provides special syntax for launching kernels on GPU
        \end{itemize}
    \end{info}

    \begin{info}{GPU specific}
        \begin{itemize}
            \item the CUDA extensions define the \emph{programming model}
            \item PRO : programming model is well suited to GPU
            \item CON : GPU-specific
        \end{itemize}
    \end{info}

    \begin{info}{Extras}
        \begin{itemize}
            \item provides library/API
            \item tools for profiling and debugging
        \end{itemize}
    \end{info}
\end{frame}

