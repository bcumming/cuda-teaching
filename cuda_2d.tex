%++++++++++++++++++++++++++++++++
\cscschapter{Going 2D and 3D}
%++++++++++++++++++++++++++++++++

%%%%%%%%%%%%%%%%%%%%%%%%%%%%%%%%%%%%%%%%%%%%
\begin{frame}[fragile]{}
%%%%%%%%%%%%%%%%%%%%%%%%%%%%%%%%%%%%%%%%%%%%
   \begin{info}{Launch Configuration}
       \begin{itemize}
           \item so far we have used one-dimensional launch configurations
           \begin{itemize}
               \item threads in blocks indexed using \lst{threadIdx.x}
               \item blocks in a grid indexed using \lst{blockIdx.x}
           \end{itemize}
           \item many kernels map naturally onto 2D and 3D indexing
           \begin{itemize}
               \item e.g. matrix-matrix operations
               \item e.g. stencils
           \end{itemize}
       \end{itemize}
   \end{info}

\end{frame}

%%%%%%%%%%%%%%%%%%%%%%%%%%%%%%%%%%%%%%%%%%%%
\begin{frame}[fragile]{}
%%%%%%%%%%%%%%%%%%%%%%%%%%%%%%%%%%%%%%%%%%%%
   \begin{info}{Full Launch Configuration}
        kernel launch dimensions can be specified with \lst{dim3} structs
        \begin{center}
        \lst{kernel@<<<@dim3 gridDim, dim3 blockDim@>>>@(...);}
        \end{center}
       \begin{itemize}
           \item \lst{dim3.x}, \lst{dim3.y} and \lst{dim3.z} specify the launch dimensions
           \item can be constructed with 1, 2 or 3 dimensions
           \item unspecified \lst{dim3} dimensions are set to 1
       \end{itemize}
   \end{info}
   \begin{code}{launch configuration examples}
        \begin{lstlisting}[style=boxcudatiny]
// 1D: 128x1x1 for 128 threads
dim3 a(128);
// 2D: 16x8x1  for 128 threads
dim3 b(16, 8);
// 3D: 16x8x4  for 512 threads
dim3 c(16, 8, 4);
        \end{lstlisting}
   \end{code}

\end{frame}

%%%%%%%%%%%%%%%%%%%%%%%%%%%%%%%%%%%%%%%%%%%%
\begin{frame}[fragile]{}
%%%%%%%%%%%%%%%%%%%%%%%%%%%%%%%%%%%%%%%%%%%%

    The \lst{threadIdx}, \lst{blockDim}, \lst{blockIdx} and \lst{gridDim} can be treated like 3D vectors via the \lst{.x}, \lst{.y} and \lst{.z} members.
    \begin{code}{matrix addition example}
        \begin{lstlisting}[style=boxcudatiny]
__global__
void MatAdd(float *A, float *B, float *C, int n) {
    int i = blockIdx.x * blockDim.x + threadIdx.x;
    int j = blockIdx.y * blockDim.y + threadIdx.y;
    if(i<n && j<n) {
        auto pos = i + j*n;
        C[pos] = A[pos] + B[pos];
    }
}
int main() {
    // ...
    dim3 threadsPerBlock(16, 16);
    dim3 numBlocks(n / threadsPerBlock.x, n / threadsPerBlock.y);
    MatAdd<<<numBlocks, threadsPerBlock>>>(A, B, C);
    // ...
}
        \end{lstlisting}
   \end{code}

\end{frame}

%%%%%%%%%%%%%%%%%%%%%%%%%%%%%%%%%%%%%%%%%%%%
\begin{frame}[fragile]{Exercise: Launch Configurations}
%%%%%%%%%%%%%%%%%%%%%%%%%%%%%%%%%%%%%%%%%%%%
    \begin{itemize}
        \item 2D stencil in \lst{diffusion/diffusion2d.cu}
        \begin{itemize}
            \item a plotting script is provided for visualizing the results
            \item use a small domain for visualization
        \end{itemize}
        \item \emph{extra}: can you improve performance with shared memory?
    \end{itemize}

    \begin{code}{running the code}
        \begin{lstlisting}[style=boxcudatiny]
module load python/2.7.6
cd cuda/practicals/diffusion

make
srun diffusion2d.cuda 8 1000000
python plotting.py
        \end{lstlisting}
   \end{code}

\end{frame}

%++++++++++++++++++++++++++++++++
\cscschapter{MPI with GPUs}
%++++++++++++++++++++++++++++++++

%%%%%%%%%%%%%%%%%%%%%%%%%%%%%%%%%%%%%%%%%%%%
\begin{frame}[fragile]{}
%%%%%%%%%%%%%%%%%%%%%%%%%%%%%%%%%%%%%%%%%%%%
    \begin{info}{MPI with data in device memory}
        Our GPU-accelerated applications use GPUs to parallelize on-node computation
        \begin{itemize}
            \item and MPI for communication between nodes
        \end{itemize}
        It is likely that communication between MPI ranks will involve information on the GPU
        \begin{enumerate}
            \item allocate buffers in host memory
            \item manually copy from device$\rightarrow$host memory
            \item perform MPI communication with host buffers
            \item copy received data from host$\rightarrow$device memory
        \end{enumerate}
        This approach can be very fast:
        \begin{itemize}
            \item have a CPU thread dedicated to asynchronous host$\leftrightarrow$device and MPI communication
        \end{itemize}
    \end{info}

\end{frame}

%%%%%%%%%%%%%%%%%%%%%%%%%%%%%%%%%%%%%%%%%%%%
\begin{frame}[fragile]{}
%%%%%%%%%%%%%%%%%%%%%%%%%%%%%%%%%%%%%%%%%%%%
    \begin{info}{GPU-Aware MPI}
        GPU-aware MPI implementations can automatically handle MPI transactions with pointers to GPU memory
        \begin{itemize}
            \item MVAPICH 2.0
            \item OpenMPI since version 1.7.0
            \item Cray MPI
        \end{itemize}
    \end{info}

    \begin{info}{How it works}
        \begin{itemize}
            \item each pointer passed to MPI is checked to see if it is in host or device memory
            if not set, MPI assumes that all pointers are to host memory, and your application will probably crash with segmentation faults\item small messages between GPUs (up to $\approx$8 k) are copied directly with \emph{RDMA}
            \item larger messages are \emph{pipelined} via host memory
        \end{itemize}
    \end{info}

\end{frame}

%%%%%%%%%%%%%%%%%%%%%%%%%%%%%%%%%%%%%%%%%%%%
\begin{frame}[fragile]{}
%%%%%%%%%%%%%%%%%%%%%%%%%%%%%%%%%%%%%%%%%%%%
    \begin{info}{How to use G2G communication}
        \begin{itemize}
            \item set the environment variable \lst{export MPICH_RDMA_ENABLED_CUDA=1}
            \begin{itemize}
                \item if not set, MPI assumes that all pointers are to host memory, and your application will probably crash with segmentation faults
            \end{itemize}
            \item experiment with the environment variable \lst{MPICH_G2G_PIPELINE}
            \begin{itemize}
                \item sets the maximum number of 512 kB message chunks that can be in flight (default 16)
            \end{itemize}
        \end{itemize}
    \end{info}
   \begin{code}{MPI with G2G example}
        \begin{lstlisting}[style=boxcudatiny]
MPI_Request srequest, rrequest;
auto send_data = malloc_device<double>(100);
auto recv_data = malloc_device<double>(100);

// call MPI with GPU pointers
MPI_Irecv(recv_data, 100, MPI_DOUBLE, source, tag, MPI_COMM_WORLD, &rrequest);
MPI_Isend(send_data, 100, MPI_DOUBLE, target, tag, MPI_COMM_WORLD, &srequest);
        \end{lstlisting}
   \end{code}

\end{frame}

%%%%%%%%%%%%%%%%%%%%%%%%%%%%%%%%%%%%%%%%%%%%
\begin{frame}[fragile]{}
%%%%%%%%%%%%%%%%%%%%%%%%%%%%%%%%%%%%%%%%%%%%
    \begin{info}{Capabilities and Limitations}
        \begin{itemize}
            \item support for most MPI API calls (point-to-point, collectives, etc)
            \item robust support for common MPI API calls
            \begin{itemize}
                \item i.e. point-to-point operations 
            \end{itemize}
            \item no support for user-defined MPI data types
        \end{itemize}
    \end{info}
\end{frame}

%%%%%%%%%%%%%%%%%%%%%%%%%%%%%%%%%%%%%%%%%%%%
\begin{frame}[fragile]{Exercise: MPI with G2G}
%%%%%%%%%%%%%%%%%%%%%%%%%%%%%%%%%%%%%%%%%%%%
    \begin{itemize}
        \item 2D stencil with MPI in \lst{diffusion/diffusion2d_mpi.cu}
        \item copy the kernel and kernel lanuch from previous example
        \begin{enumerate}
            \item implement MPI communication with host buffering
            \item implement MPI communication with G2G
            \item can you observe any performance differences between the two?
            \item why are we restricted to just 1 MPI rank per node?
        \end{enumerate}
    \end{itemize}

    \begin{code}{running the MPI example}
        \begin{lstlisting}[style=boxcudatiny]
cd cuda/practicals/diffusion
make
srun -n2 -N2 diffusion2d_mpi.cuda 8
module load python/2.7.6
python plotting.py
# once it gets the correct results:
sbatch job.batch
        \end{lstlisting}
   \end{code}

\end{frame}

%%%%%%%%%%%%%%%%%%%%%%%%%%%%%%%%%%%%%%%%%%%%
\begin{frame}[fragile]{Exercises: 2D Diffusion with MPI Results}
%%%%%%%%%%%%%%%%%%%%%%%%%%%%%%%%%%%%%%%%%%%%
   \begin{info}{Time for 1000 time steps \@ 128$\times$16,382}
       \begin{center}
           \begin{tabular}{ccc}
               \hline
               nodes   &   G2G off & G2G on \\
                \hline
                1      &   0.479   & 0.479  \\
                2      &   0.277   & 0.274  \\
                4      &   0.183   & 0.180  \\
                8      &   0.152   & 0.151  \\
               16      &   0.167   & 0.117  \\
           \end{tabular}
       \end{center}
   \end{info}
\end{frame}

